\documentclass[a4paper,10pt]{article}
%\usepackage[margin=2.5cm, nohead]{geometry}
\usepackage{palatino, url, multicol}
\usepackage{hyperref}
\usepackage{graphicx}
\usepackage{verbatim}
\usepackage[all]{xy}
\usepackage[english]{babel}
\usepackage{cite}


\title{Bachelor project: Assignment 3\\\vspace{.2cm}\large A First Step in Literature and
Research Goals}
\author{Maarten de Waard\\\small 5894883}
\begin{document}
\maketitle
\section{Research goal}
The research will be focussing on combining the \textit{Extensible Agent
Behavior Specification Language} (XABSL) with the current RoboRescue code, used
by Arnoud Visser, in the Robocup RoboRescue missions. 

In roborescue there has been a lot of research to make the current `behaviors'
as good as possible. Trying to improve that as a bachelor thesis would be a
futile attempt. There are however a lot of possibilities of improving how the
current behaviors are selected. XABSL is one of these possibilities, and has
proven itself in robot soccer competitions. \cite{lotzsch2004designing} \cite{loetzsch2006xabsl}

My goal is to use XABSL to make the current RoboRescue code work more efficient
and more human-readable. The first of these can be measured by comparing the
efficiency of the code. In this case that would mean comparing the efficiency of
autonomously 
exploring an unknown area by the current code, and the code using XABSL.

This type of controlling behavior has not been done in Rescue yet. The closest I
could find was about using behavior for navigation on challenging terrain.
\cite{seraji2002behavior} If I succeed early, I could start experimenting with
whatever extra improvements can be done, by changing some parameters, or the
hierarchy of the FSA's. I can then easily find out which FSA-hierarchy
structures
works better and which work worse. 

\section{Keywords and relations}
Since I don't understand how a concept map can help me specify my research, I
can only list the keywords:
\begin{itemize}
    \item XABSL
    \item Roborescue
    \item Behavior based robotics
    \item Robots
    \item Autonomous exploration
\end{itemize}

I think the relations between these keywords are trivial, and thus do not need
any explanation.

\section{Literature}
Because I have written most of this section for my LiteratureLog page on the
wiki of my googlecode
page\footnote{\url{http://code.google.com/p/crax/wiki/LiteratureLog}}, this
section is in
dutch.
\subsection*{Behavior-Based Robot Navigation on Challenging Terrain: A Fuzzy
Logic Approach\cite{seraji2002behavior}}
Algemene inhoud:
\textit{This paper presents a new strategy for behavior-based navigation of field mobile
robots on challenging terrain, using a fuzzy logic approach and a novel measure
of terrain traversability}

\begin{itemize}
\item Sectie 3 gaat over de structuur van bahavior-based navigation strategy.
\item Sectie 4, 5 en 6 leggen uit hoe hun bahavior-based navigation strategy in elkaar
zit.
\item Sectie 7 gaat over hun behavioral systeem. Niet met een FSA, maar met een iets
ander systeem
\item Sectie 8 biedt wat interessante manieren om mijn conclusie uiteindelijk te
verwoorden.
\item Sectie 9 bespreekt de resultaten van hun ondeszoek met field studies, voor mij
niet zo interessant.
\item Sectie 10 heeft de daadwerkelijke conclusie
\end{itemize}

Wat heb ik aan dit paper:
Een groot deel gaat over het navigeren op 'rough terrain'. Daar ga ik niets mee
doen, hun exacte behavior (sectie 4, 5, 6) is ook niet heel interessant. De
manier waarop ze hun behavior kiezen (sectie 8) wel, en de manier waarop ze
rechtvaardigen dat dit interessant is voor de wereld (sectie 8 en 10) ook.

\subsection*{XABSL - A Pragmatic Approach to Behavior
Engineering\cite{loetzsch2006xabsl}}

Algemene inhoud: Het paper legt uit hoe XABSL toegepast kan worden als
`hierarchie van finite state machines'.
\begin{itemize}
\item Sectie 2 introduceert de architectuur van XABSL
\item Sectie 3 omschrijft de taal en architectuur
\item Sectie 4 omschrijft de applicaties van XABSL in verschillende domeinen.
\end{itemize}

Wat heb ik aan dit paper: Algemene introductie, misschien referentie naar
gebruik van de taal.

\subsection*{Designing Agent Behavior with the Extensible Agent Behavior
Specification
Language XABSL\cite{lotzsch2004designing}}

Algemene inhoud:
Legt uit hoe XABSL wordt gebruikt in de robocup 4-legged voetbal league

\begin{itemize}
\item Sectie 2 omschrijft de ontwikkeling in XABSL
\begin{itemize}
    \item Sectie 2.1 biedt daarin een korte introductie
    \item Sectie 2.2 omschrijft de XML-taal binnen het XABSL-framework
    \item Sectie 2.3 omschrijft het Runtime-systeem, de XabslEngine
\end{itemize}
\item Sectie 3 omschrijft een applicatie
\item Sectie 4 biedt een korte conclusie
\end{itemize}

Wat heb ik aan dit artikel: Het artikel geeft een duidelijke introductie in het
werken met XABSL.


\subsection*{Beyond frontier exploration\cite{visser2008beyond}}

Algemene inhoud: Het paper legt uit hoe de huidige code werkt, vooral hoe de
SLAM code werkt.

Wat heb ik aan dit paper: introductie van de code.    

\subsection*{5e artikel}
Hier hoort voor een goed afgemaakte opdracht nog een 5e artikel. Deze heb ik nog
niet gevonden, omdat ik nog niet zo veel met het framework heb gewerkt, en
daardoor nog geen exacte richting heb gevonden om naar het volgende paper te
zoeken. Er zullen ongetwijfeld tijdens het onderzoek een hoop vragen moeten
worden beantwoord, waarvoor ik extra artikelen nodig krijg. Een alternatief voor
een artikel hier was meer van hetzelfde geweest. Zo ga ik een hoop informatie
halen uit \cite{dehoog2011role} en \cite{balch1998behavior}, maar deze
boekwerken van ruim 200 pagina's heb ik nog niet volledig gelezen, en ik weet
nog niet exact wat ik eraan heb.
 

\bibliography{bib}{}
\bibliographystyle{plain}
\end{document}

