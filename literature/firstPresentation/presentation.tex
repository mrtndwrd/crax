\documentclass{beamer}
\usepackage[utf8x]{inputenc}
%\usepackage{default}
\usepackage{verbatim}

\title{Combining Roborescue and XABSL}
\author{Maarten de Waard}
\institute{UvA}
\usetheme{Berkeley}
\newcommand{\slide}[2]
{
\begin{frame}
\frametitle{#1} 

#2

\end{frame}
}



\begin{document}
\begin{frame}
\titlepage
\end{frame}



\section{Used projects}
\slide{Roborescue}
{
    Currently roborescue mainly involves the exploration task:
\vspace{0.5cm}
\begin{itemize}
	\item Deploy several robots and a base station
	\item Autonomous exploration will be started:
	\begin{itemize}
		\item Simultaneous Localization and Mapping
		\item Avoiding obstacles
        \item Avoiding walls
    \end{itemize}
    \item Different subroutines are used.
\end{itemize}
}
\slide{XABSL\\\small Extensible Agent Behavior Specification Language}
{
   \textit{XABSL is a very simple language to describe behaviors for autonomous agents
   based on hierarchical finite state machines.}\footnote{www.xabsl.de}
    
    Advantages:
    \begin{itemize}
        \item Simple to use
        \item Easy to keep track on a big FSM hierarchy
        \item Lots of debugging tools
        \item Lots of testing tools
    \end{itemize}
}
\section{Research goal}
\slide{Combining the two}
{
Why:
\begin{itemize}
\item XABSL has proven itself in the robocup soccer competition
\item Behavior has not been used in rescue (and exploration) yet
\item The current rescue code is in Visual Basic: 
    \begin{itemize}
    \item It's difficult to organize a big behavior hierarchy in Visual Basic
    \item It's easy to lose track of what you are doing
    \end{itemize}
\end{itemize}
}

\section{Measuring results}
\slide{Measuring Results}
{
    How results can be measured:
    \begin{itemize}
        \item Ammount of exploration done
        \item Time taken for current ammount of exploration
    \end{itemize}
    These should be compared with the results of the old code. Furthermore it
    should be easier to tweak variables, which could give extra advantages in
    the points above.
}

\end{document}
